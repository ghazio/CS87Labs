\documentclass[11pt]{article}
\usepackage{fullpage}

% this starts the document
\begin{document}

\title{CS87 Spring 18:  Lab 5 Answers to Questions}

\author{Student1, Student2 }

\maketitle

{\bf Question 1:} What is the big-O complexity of the Sequential Odd-Even Sort
Algorithm on a list of N items?  Give a detailed explain of you answer.


\vspace{0.5cm}

{\bf Question 2:} Given P processors, what is the big-O complexity of the
Parallel Odd-Even Sorting Algorithm?  Give a detailed explain of you answer.


\vspace{0.5cm}

{\bf Question 3:} Given P processors, how much space is needed to perform the
parallel sort of N values? Explain your answer.

\vspace{0.5cm}

{\bf Question 4:} IS Odd-Even sort a good algorithm for sorting on a GPU? Why or
Why not?

\vspace{0.5cm}

{\bf Question 5:} Is Odd-Even sort a good algorithm for sorting on a cluster
using MPI? Why or Why not?



%% -----
%% example of verbatim for code sequences:
%% -----
%\begin{verbatim}
%for(j=0; j < M; j++) {
%  C[i][j] = A[i][j] + B[i][j];
%}
%\end{verbatim}

%% -----
%% math mode
%% -----
% blocks \begin{math}   \end{math}
% inline: $ <interpreted in math mode> $
% ex:
%\begin{math}
%\forall x \in X, \quad \exists y \leq \epsilon
%\end{math}
% For example, $x_{i} = y^{3} + 2x_{i-1}$, is an inlined math form.


\end{document}
