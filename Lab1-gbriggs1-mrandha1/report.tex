\documentclass[11pt]{article}
\usepackage{fullpage}
\usepackage{subfigure,indentfirst}
% for url
\usepackage{hyperref}
% for underlined text
\usepackage[normalem]{ulem}
% a package for importing pdf figures, there are other graphics
% packages you can use for different file types, but sadly they
% are not all compatible, so you often have to convert figures 
% from one type to another to use a particular package
\usepackage{graphicx}

% this starts the document
\begin{document}

\title{CS87:F21 Lab 1 Report}

\author{Student1, Student2 }

\maketitle

% the structure of your report should be
\section {Introduction}   

\section {Experiement Descriptions}

\subsection{Here is how to create a subsection if you want}

\section {Results}

\section {Conclusions}

%% example latex formatting for figures, lists, tables, math, verbatim
% (in /home/newhall/public/latex_examples/report/ are more examples)
% Uncomment (remove leading %) and run make report to try them out
% and cut and paste as starting points in your report

%% -----
%% figure: including a .pdf figure
%% -----
\begin{figure}[t]
\centerline{\includegraphics[height=3.0in]{/home/newhall/public/latex_examples/report/ERbook.pdf}}
\caption{ {\label{bookER} Figures should have a caption that 
    briefly describe what the figure is/shows.  
}}
\end{figure}

%You can refer to a figure in prose by adding a label to the figure,
%and the ref'in it, like this: Figure~\ref{bookER}.  

%% -----
%% lists:  enumerate or itemize
%% -----
% \begin{enumerate}
% \item first item in list
% \item next item ...
% \end{enumerate}


%% -----
%% table: with caption
%% -----
%\begin{table}
%\begin{center}
%\begin{tabular}{|l||r l|c|}
%\hline
%  {\bf Benchmark} & {\bf Size} & {\bf Num} & {\bf Time (secs)} \\
%\hline
%  Radix &  2024  &  32  &  255.5 \\
%  HPL   &  2024  &  32  &  815.7 \\
%\hline
%\end{tabular}
%\caption{\label{results1} A 1-2 sentece high-level summary of this table: 
%  what it is and what main result it shows.  ex: "Radix and HPL Scalability".
%  Radix scales better with problem size"
%  {\em optional details about interpreting fields ex: "For each benchmark, 
%  the total run time (in seconds) when run using Nswap2L with prefetching 
%  enabled."  
%    } }
%\end{center}
%\end{table}

%% Notes:
%% The first part of a tabular definition {\tt e.g. |c|l|r|rr|} 
%% specifies the number of columns, the text alignment in each column,
%% (c:centered, l:left, r:right), and (|) any vertical bars between columns.
%% Each row of values is listed on a separate line 
%% Ampersands are used between each column's value.
%% \\ ends a row.   
%% \hline can be used to draw horizontal lines.  

%% -----
%% verbatim for code sequences:
%% -----
\begin{verbatim} 
for(j=0; j < M; j++) {
  C[i][j] = A[i][j] + B[i][j];
}
\end{verbatim}

%% -----
%% math mode
%% -----
% blocks \begin{math}   \end{math}
% inline: $ <interpreted in math mode> $ 
% ex:
%\begin{math}
%\forall x \in X, \quad \exists y \leq \epsilon 
%\end{math}
% For example, $x_{i} = y^{3} + 2x_{i-1}$, is an inlined math form.


\end{document}

